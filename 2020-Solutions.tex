\documentclass{amsart}

\setlength{\parindent}{0pt}
\setlength{\parskip}{\baselineskip}

\raggedbottom

\begin{document}
\begin{center}
Science Atlantic Math Competition 2020 Solutions
\end{center}

\textbf{Problem 1}\\
Write $\tan(x) + \cot(2x)$ in the form $bf(cx)$, where $b$ and $c$ are real numbers, and $f$ is a standard trigonometric function.

\textit{Solution}\\
\begin{align*}
\tan(x) + \cot(2x) &= \frac{\sin(x)}{\cos(x)} + \frac{\cos(2x)}{\sin(2x)} \\
&= \frac{\sin(x)}{\cos(x)} + \frac{\cos^2(x) - \sin^2(x)}{2\sin(x)\cos(x)} \\
&= \frac{2 \sin^2(x) + \cos^2(x) - \sin^2(x)}{2\sin(x)\cos(x)} \\
&= \frac{\sin^2(x) + \cos^2(x)}{2\sin(x)\cos(x)} \\
&= \frac{1}{\sin(2x)} = \csc(2x)
\end{align*}

\pagebreak

\textbf{Problem 2}\\
A bug located at $(2, 0, 0)$ in $\mathbb{R}^3$ wants to get to $(-2, 0, 0)$ but is impeded by an impenetrable sphere of radius $1$ centred at the origin.
Describe and give the length of the shortest path available to this bug.

\textit{Solution}\\
The projection of a path onto a plane is no longer than the path, so with no loss of generality, model the problem as the shortest path from $(2, 0)$ to $(-2, 0)$ which does not pass through the unit circle.
If a string is stretched from $(2, 0)$ to $(-2, 0)$, its path consists of two line segments from $(-2, 0)$ and $(2, 0)$ meeting the unit circle at the points of tangency and the arc between those points.
The points of tangency lie at $(-1/2, \sqrt{3}/2)$ and $(1/2, \sqrt{3}/2)$, or at angles of $2\pi/3$ and $\pi/3$ on the unit circle, respectively.
The distances from $(-2, 0)$ to $(-1/2, \sqrt{3}/2)$ and from $(2, 0)$ to $(1/2, \sqrt{3}/2)$ are both $\sqrt{3}$, and the length of the arc between $\pi/3$ and $2\pi/3$ is $\pi/3$, so the length of the path is $2\sqrt{3} + \pi/3$.

\pagebreak

\textbf{Problem 3}\\
A tetrahedron has vertices $ABCD$.
Suppose that the plane $\gamma$ bisects the (internal) dihedral angle along edge $AB$ and meets edge $CD$ at $G$.
Prove that
\[ \frac{\lvert ABC \rvert}{\lvert ABD \rvert} = \frac{\lvert CG \rvert}{\lvert GD \rvert} \]

\textit{Solution}\\

\pagebreak

\textbf{Problem 4}\\
A sequence of natural numbers is \textit{eccentric} if no term can be written as a sum of terms that come before it (repetition is allowed).
For instance, the finite sequence $12, 3, 13, 17, 2$ is eccentric, but $11, 3, 13, 17, 2$ is not because $11 + 3 + 3 = 17$.
Does there exist an infinite eccentric sequence?

\textit{Solution}\\
Let $(a_n)_{n = 0}^\infty$ be a sequence of natural numbers.
If any terms are repeated, the sequence is not eccentric (because a single term is considered a sum).

Suppose no terms are repeated.
By the pigeonhole principle, there exists $b$ such that $0 \leq b < a_0$ and infinitely many $a_n \equiv b \pmod{a_0}$.
Then there exists a subsequence $(a_{n_i})_{i = 0}^\infty$ of $(a_n)$ such that each $a_{n_i} \equiv b \pmod{a_0}$.
Then there exists $j$ such that $a_{n_0} = ja_0 + b$.
Since $(a_n)$ has no repetitions, there exists $k \leq j + 1$ such that $a_{n_k} > a_{n_0}$, implying that there exists $\ell > j$ such that $a_{n_i} = \ell a_0 + b$.
Then $a_{n_i} = a_{n_0} + (\ell - j) a_0$, so $(a_n)$ is not eccentric.

Therefore no infinite eccentric sequences exist.

\pagebreak

\textbf{Problem 5}\\
Define a \textit{factorial $M$-partition} of $N$ to be a set $\{a_1, a_2, \cdots, a_M\}$ of positive integers such that $\displaystyle N = \sum_{i = 1}^M a_i!$.
Furthemore, define a factorial $M$-partition to be \textit{proper} if the $a_i$ are all unequal.
Show that if for some $M, N$ there is a proper factorial $M$-partition of $N$, then every other proper factorial $M$-partition of $N$ is a permutation of it.

\textit{Solution}\\
First we show that $1! + \cdots + (n - 1)! < n!$ by induction.
Indeed, $1! < 2!$ and if $n > 1$ and $1 + \cdots + (n - 1)! < n!$, then $1 + \cdots n! < 2n! < (n + 1)n! < (n + 1)!$.

Now we show the main result by induction on $M$.
For $M = 1$, $\{a!\} = \{b!\} \iff a = b$.
Now suppose for a given $M$ that for every $N$ with a proper factorial $M$-partition, $N$ admits only one such partition up to ordering.
Let $N$ be a number with a proper factorial $(M + 1)$-partition.
Let $n$ be the unique natural number such that $n! \leq N < (n + 1)!$.
Since $1! + \cdots + (n - 1)! < n! \leq N$, at least one element of the partition is $\geq n$.
On the other hand, all terms of the partition are $<(n + 1)$ so the largest element is $n$.
Now by supposition, $N - n$ has a unique ordered proper factorial $M$-partition, so appending $n$ to this gives a unique ordered proper factorial $(M + 1)$-partition of $N$.

Therefore the result holds for all $M, N$ by induction.

\pagebreak

\textbf{Problem 6}\\
Show for integers $\geq 1$ that
\[ \frac{d^n}{dx^n} (1 + x)^{n - 1} e^\frac{x}{x + 1} = \frac{e^\frac{x}{x + 1}}{(1 + x)^{n + 1}} \]

\textit{Solution}\\
Let $y = 1 + x$ and note that $\displaystyle (1 + x)^{n - 1} e^\frac{x}{x + 1} = y^{n - 1} e^{1 - \frac{1}{y}}$.
For each $m, n$ let $\displaystyle f_n^{(m)} = \frac{d^m}{dx^m} y^{n - 1} e^{1 - \frac{1}{y}}$.

First we show by induction on $m$ that $f_{(n + 1)}^{(m)} = mf_n^{(m - 1)} + yf_n^{(m)}$.
For the base case $m = 1$, $\displaystyle f_{(n + 1)}^{(1)} = \frac{d}{dx} yf_n = f_n + yf_n^{(1)}$.
For the step case, if $f_{(n + 1)}^{(m)} = mf_n^{(m - 1)} + yf_n^{(m)}$, then $f_{(n + 1)}^{(m + 1)} = mf_n^{(m)} + f_n^{(m)} + yf_n^{(m + 1)} = (m + 1)f_n^{(m)} + yf_n^{(m + 1)}$.

Now we show by induction on $n$ that $f_n^{(n)} = y^{-(n + 1)} e^{1 - \frac{1}{y}}$ as desired.
The base case $n = 0$ is obvious.
For the step case, $f_{(n + 1)}^{(n)} = nf_n^{(n - 1)} + yf_n^{(n)}$ and so $f_{(n + 1)}^{(n + 1)} = nf_n^{(n)} + f_n^{(n)} + yf_n^{(n + 1)} = (n + 1)f_n^{(n)} + yf_n^{(n + 1)}$.
By supposition, $f_n^{(n)} = y^{-(n + 1)} e^{1 - \frac{1}{y}}$ and so $f_n^{(n + 1)} = (-(n + 1)y^{-(n + 2)} + y^{-(n + 3)}) e^{1 - \frac{1}{y}}$.
Substituting, we have $f_{n + 1}^{(n + 1)} = y^{-(n + 2)} e^{1 - \frac{1}{y}}$ as desired.

\pagebreak

\textbf{Problem 7}\\
Assess the convergence of
\[ \sum_{n = 0}^\infty \arctan\left(\frac{2n - 1}{n^4 - 2n^3 + n^2 + 1}\right) \]
If convergent, give the value of the series.

\textit{Solution}\\
By the rule
\[ \displaystyle \tan(x - y) = \frac{\tan(x) - \tan(y)}{1 + \tan(x)\tan(y)} \]
we have that
\[ \displaystyle \arctan\left(\frac{2n - 1}{n^4 - 2n^3 + n^2 + 1}\right) = \arctan\left(\frac{n^2 - (n - 1)^2}{1 + n^2(n - 1)^2}\right) = \arctan(n^2) - \arctan((n - 1)^2) \]
Then
\begin{align*} \sum_{n = 0}^\infty \arctan\left(\frac{2n - 1}{n^4 - 2n^3 + n^2 + 1}\right) &= \sum_{n = 0}^\infty \arctan(n^2) - \arctan((n - 1)^2) \\
&= -\arctan((-1)^2) + \lim_{n \to \infty} \arctan(n^2) \\
&= -\frac{\pi}{4} + \frac{\pi}{2} = \frac{\pi}{4}
\end{align*}

\pagebreak

\textbf{Problem 8}\\
The symmetric $n \times n$ matrix $A_n$ has $ij$-th entry
\[ \begin{cases} (3 - (-1)^i) / 2 & i = j \\ -1 & i = j \pm 1 \\ 0 & \text{otherwise} \end{cases} \]
Determine $\det(A_n)$ for all positive integers $n$.

\textit{Solution}\\
Let $e_i = (3 - (-1)^i)/2$.
An easy calculation shows that $A_1 = [2]$ and $A_2 = \begin{bmatrix} 2 & -1 \\ -1 & 1 \end{bmatrix}$ have determinants 2 and 1 respectively.
In the general case, $A_n$ has $A_{n - 2}$ as the top left $(n - 2) \times (n - 2)$ sub-block and $A_{n - 1}$ as the top left $(n - 1) \times (n - 1)$ sub-block.
\[ \begin{bmatrix}
& & & \cdots & \cdots \\
& A_{n - 2} & & \cdots & \cdots \\
& & & -1 & \cdots \\
\cdots & \cdots & -1 & e_{n - 1} & -1 \\
\cdots & \cdots & \cdots & -1 & e_n
\end{bmatrix} \]
Then $\det(A_n) = e_n \det(A_{n - 1} - \det(A_{n - 2})$ by evaluating first along the $n$-th row, then the $n$-th column of the resulting minors.
This recurrence yields $\det(A_3) = 0$, $\det(A_4) = -1$, $\det(A_5) = -2$, $\det(A_6) = -1$, $\det(A_7) = 0$, $\det(A_8) = 1$, $\det(A_9) = 2$, and $\det(A_{10}) = 1$.
Since $\det(A_9) = \det(A_1)$ and $\det(A_{10}) = \det(A_2)$, the sequence of determinants has period 8 and follows this pattern.

\end{document}